% Auto-generated from DOCX
\section{Conclusion}
We have presented a framework for applying public announcement logic (PAL) and related epistemic logic principles to manage the knowledge state of LLM-based agents. By representing the agent's knowledge as a minimal set of facts (announcements) and updating this set whenever new information is obtained, the agent effectively simulates the elimination of possible worlds without ever enumerating them. This approach brings several advantages: the agent maintains consistency, can explicitly reason about what it knows or doesn't know, and can plan information-gathering actions accordingly. We grounded key epistemic concepts, including possible worlds, knowledge modalities, common knowledge, and announcements, in the context of LLM agent workflows, illustrating with examples how an agent's log of observations serves as an epistemic filter on its beliefs\cite{espf}. We also discussed extensions to graded and fuzzy modal logics, acknowledging that real-world agents operate under uncertainty and may assign degrees of belief rather than certainties\cite{del-sep}. While our current implementation treats incoming information as truthful knowledge updates (public announcements in the strict sense\cite{del-sep}), the framework is flexible enough to accommodate belief updates with less than full certainty, using plausibility or probability-based reasoning from dynamic doxastic logic\cite{del-sep}.

Our framework essentially functions as a protocol layer for structured cognition in LLM agents, independent of the underlying language model. It interfaces with the LLM by feeding it the up-to-date knowledge state and intercepting its outputs for new potential facts. This creates a positive feedback loop: as the LLM derives or discovers information, the state is updated, which in turn informs subsequent reasoning. We showed that this approach can be integrated into existing agent architectures that use chain-of-thought prompting and tool use, with relatively low overhead. The payoff is that the agent avoids many pitfalls such as forgetting earlier information, contradicting itself, or blindly guessing in the face of unknowns. Especially as tasks become long or collaborative, having an explicit notion of ``what we (agent or team) know so far'' becomes crucial. Our approach offers an implementable solution to that, inspired by decades of research in epistemic logic and multi-agent systems.

There are, of course, challenges and open questions remaining. We need to further develop methods for belief revision when the agent encounters contradictions, possibly leveraging the LLM's reasoning to decide which prior assumptions to drop. We also should explore how the framework scales in truly open-ended environments, where new propositions can continuously appear. The worry of unbounded proposition space is mitigated by the agent's focus on relevant info, but formal assurances (e.g. complexity analysis, or constraints on $\Sigma$ growth) would strengthen the approach. On the evaluation front, we propose constructing benchmarks that specifically test an agent's ability to update and use its knowledge state, such as puzzle-solving tasks, or simulated exploration tasks where remembering visited locations matters. By comparing an agent with and without our epistemic layer, we can quantify improvements in correctness and efficiency (e.g., fewer redundant actions, higher success rate on tasks requiring memory of past clues).

In terms of impact and applicability, this framework could enhance any AI system that requires a mix of learning and reasoning. A few examples include: virtual assistants that must keep track of user preferences across sessions, autonomous scientific discovery systems that accumulate findings and must avoid contradicting prior evidence, and multi-agent teams (human-AI or AI-AI) where maintaining a shared knowledge base is key to coordination. By formalizing the knowledge updates, we provide a clear interface for trust: a human supervisor could inspect the knowledge log $\Sigma$ to verify if the agent is operating on correct information. This is an advantage over end-to-end neural methods where the internal state is a opaque vector.

Finally, we note that our work sits at an intersection of symbolic and neural methods, and exemplifies how insights from logic can help tackle modern AI problems. The use of PAL, a relatively niche logical framework, in a cutting-edge LLM scenario is a testament to the enduring relevance of symbolic reasoning for AI's explainability and reliability. As LLMs become more capable, ensuring they have robust internal states and know when to seek information will be increasingly important. We believe the ideas presented here are a step toward LLM agents that are not just linguistically competent, but epistemically self-aware.



